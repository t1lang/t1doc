\section{Lexing}\label{lexing}

We describe here how the T1 interpreter breaks down source code into
individual tokens, upon which the T1 syntax is built. Since T1 is
generically extensible (interpreted source code can, at any point,
invoke itself and take over processing of the remaining of the source
code), an arbitrary number of new parsing rules can be implemented by
source code. The rules detailed below explain how parsing is done at the
start of the source code processing.

\begin{rationale}
In Forth, the only lexing process is to aggregate sequences of non-space
characters into ``words''. All other syntax, e.g. literal strings or
comments, is implemented by custom ``immediate words'', which are
functions that are invoked right away when encountered in the source
stream. T1 implements a more complicated lexing process for ease of
development.

It shall be noted that Forth aims at offering support for development
right on the target system, which may be embedded and constrained; this
is one of the main reason for the very simple lexing process of Forth.
In T1, the normal model is to make development on a dedicated powerful
development workstation, distinct from the target system on which the
code will run, which is why more expensive lexing is not an issue.
\end{rationale}

\paragraph{Input Characters.} Source code consists in a number of text
streams that are processed in due order. They are normally stored as
individual files. Each stream contains bytes which are interpreted into
\emph{characters} using UTF-8 encoding; in this specification, a
character is a Unicode ``code point'', i.e. an integer in the
\verb|U+0000| to \verb|U+10FFFD| range. Outside of literal strings, only
ASCII characters (\verb|U+0000| to \verb|U+007E|) may appear.

\begin{rationale}

As will be described below, all the lexing really operates on bytes. In
UTF-8 encoding, each ASCII character is encoded as a single byte with
the same value, and all other code points are encoded as sequences of
bytes of value \verb|0x80| or more. The non-ASCII byte values that
appear in literal strings can thus be simply copied, since, as we shall
see, string values are really arrays of bytes. Moreover, the source
itself can define and then invoke functions that can take over source
code processing in arbitrary ways, and such functions may interpret
source bytes differently. In that sense, it is not strictly necessary
that source file uses UTF-8 encoding, only that the parts that rely on
the lexing process described here use only ASCII characters outside of
literal strings.

However, it is expected that most source code writing will be done with
text editors, that are likely to rely on, and enforce, a specific
encoding charset. In the interest of maximum interoperability, it is
here defined that all source code \emph{shall} be UTF-8 encoded, and
interpreters/compilers may enforce it.
\end{rationale}

Text breaks down into lines; each line is terminated by a newline
character (\verb|U+000A|). If a line ends with a CR+LF sequence
(\verb|U+000D| followed by \verb|U+000A|), then this is considered to be
equivalent to a single newline character.

\paragraph{Whitespace.} \emph{Whitespace} is any sequence of one or more
characters in the \verb|U+0000| to \verb|U+0020| range, i.e. all ASCII
control characters, and the ASCII space. Thus, tabulations
(\verb|U+0009|) and newline (\verb|U+000A|) are whitespace. Whitespace
characters separate tokens, but are otherwise not significant.
Indentation, in particular, is a purely aesthetic choice with no impact
on semantics. Take care that whitespace characters appearing within
literal strings do not count as whitespace.

\paragraph{Comments.} The character ``\verb|#|'' starts a comment
(unless it appears within a literal string or a character constant). The
comment spans to the end of the current line, but does not include the
newline character that terminates that line. If a comment appears on the
last line of a file that does not end with a newline character, the
comment spans to the end of the file. Comments are ignored; since, in
general, a comment is immediately followed by a newline character, that
newline character acts as whitespace.

\paragraph{Single-Character Tokens.} Each of the following characters,
when encountered outside of a literal string or character constant, counts
as a token in its own right:
\begin{verbatim}
    ( ) [ ] { } '
\end{verbatim}

\paragraph{Names and Numerical Constants.} The lexer parses a
\emph{word} as a sequence of non-space printable ASCII characters
(\verb|U+0021| to \verb|U+007E|), excluding the following characters:
\begin{verbatim}
    ( ) [ ] { } ' " #
\end{verbatim}
The lexing process is ``greedy'': the longest sequence of allowed
characters is assembled, and stops at the first disallowed character
(from the list above), whitespace, or end-of-stream, whichever comes
first.

\emph{Numerical Constants} include the following:
\begin{itemize}

    \item \emph{Boolean constants} are the words ``\verb|true|'' and
    ``\verb|false|''.

    \item \emph{Character constants} are all words that start with a
    backquote character (``\verb|`|'', \verb|U+0060|).

    \item \emph{Number constants} are all the words that start with an
    ASCII digit (``\verb|0|'' to ``\verb|9|''), or a minus
    (``\verb|-|'') or plus sign (``\verb|+|'') followed by an ASCII
    digit.

\end{itemize}

\emph{Names} are words which are not numerical constants.

Note that a word which starts with a sequence that introduces a numerical
constant, but fails to parse as a valid numerical constant, triggers an
error; it is not ``demoted'' to being a name.

\begin{rationale}
In Forth, when a word is encountered, it is first evaluated as a
function name; this works because Forth uses a strict define-before-use
policy, so any word can be unambiguously matched against existing
functions at this point. Only words which are not recognized as function
names will be re-interpreted as possible numerical constants. A side
effect is that it allows defining functions with names such as
\verb+42+, a feature which is more confusing than useful.

In T1, we allow referencing functions and types that will be defined
later on, and thus we cannot use numerical interpretation as a fallback
for unknown function names. The non-reinterpretation of invalid
numerical constants as names is meant to promote readability: looking at
the start of a word is enough to know whether it is a numerical constant
or a name; it also allows later versions of T1 to enrich the syntax with
more numerical constants, e.g. floating-point values, without breaking
backward compatibility.

A consequence is that function names cannot start with a digit, such as
Forth's ``\verb+2DUP+''.
\end{rationale}

\paragraph{Number Constants.} Valid number constants are:
\begin{itemize}

    \item a sequence of ASCII digits, interpreted as an integer value
    in base 10;

    \item the sequence ``\verb|0x|'' or ``\verb|0X|'', followed by
    one or more hexadecimal digits (hexadecimal digits are ASCII digits
    ``\verb|0|'' to ``\verb|9|'', ASCII uppercase letters
    ``\verb|A|'' to ``\verb|F|'', and ASCII lowercase letters
    ``\verb|a|'' to ``\verb|f|''), interpreted as an integer value
    in base 16 (case is not significant);

    \item the sequence ``\verb|0b|'' or ``\verb|0B|'', followed by one
    or more binary digits (``\verb|0|'' or ``\verb|1|''), interpreted as
    an integer value in base 2;

    \item any of the above, preceded by a minus sign (``\verb|-|'') or a
    plus sign (``\verb|+|''); the minus sign makes the value negative,
    while the plus sign does not change the value and is purely
    cosmetic;

    \item any of the above, followed by a suffix in the following list,
    and defining the constant to have the corresponding modular integer
    type: \verb|i8| \ \verb|i16| \ \verb|i32| \ \verb|i64| \ \verb|u8|
    \ \verb|u16| \ \verb|u32| \ \verb|u64|

\end{itemize}

If the number constant does not have an explicit type suffix, then it
has plain integer type (\verb|std::int|, as will be defined in
section~\ref{types}). If the value does not fit in the allowed range for
the target type, then an error is raised.

\paragraph{Character Constants.} A \emph{character constant} describes
an integer value of type \verb|std::u8| in the 0 to 126 range
(inclusive). Valid character constants consist in a backquote character
(\verb|U+0060|) followed by:
\begin{itemize}

    \item a single ASCII character in the \verb|U+0021| to
    \verb|U+007E| range, excluding the backslash character
    (``\verb|\|'');

    \item an escape sequence that starts with a backslash, followed
    by one character:
    \begin{itemize}

        \item \verb|\s| stands for space (\verb|U+0020|);
        \item \verb|\t| stands for tabulation (\verb|U+0009|);
        \item \verb|\r| stands for carriage return (\verb|U+000D|);
        \item \verb|\n| stands for newline (\verb|U+000A|);
        \item \verb|\'| stands for quote (\verb|U+0027|);
        \item \verb|\`| stands for backquote (\verb|U+0060|);
        \item \verb|\"| stands for double-quote (\verb|U+0022|);
        \item \verb|\\| stands for backslash (\verb|U+005C|).

    \end{itemize}

\end{itemize}

In all cases, the character constant stands for the numerical value that
corresponds to the represented code point.

\begin{rationale}
Character constants are deliberately limited to plain ASCII because
they have type \verb|std::u8|, following the decision that ``normal''
strings really are sequences of bytes. This will be explained in more
details in section~\ref{types}.
\end{rationale}

Since character constants are self-terminated (inspection of their
contents is enough to decide that no extra character follows in the
token), they need not be separated by whitespace from the next token.
Thus, ``\verb|`ab|'' is parsed as two tokens, the character constant for
lowercase letter ``a'', then the one-character name ``\verb|b|''. This
is of course confusing, so don't do that.

\paragraph{Literal Strings.} A \emph{literal string} represents a value
which is a sequence of bytes. Such a token starts with a double-quote
character ``\verb|"|'' and ends at the next unescaped double-quote
character. The following rules apply:
\begin{itemize}

    \item The starting and ending double-quote characters are not part
    of the string contents.

    \item Bytes appearing in the string literal, other than backslash
    and newline, are part of the string literal. This includes all byte
    values, even ASCII control characters (note that a CR+LF sequence at
    the end of a source text line counts as a single newline character,
    which cannot appear unescaped within a literal string).

    \item When a backslash appears within a literal string:
    \begin{itemize}

        \item If the backslash is immediately followed by the newline
        (or CR+LF) that ends the line, then this is a \emph{line
        escape}: the next line must start with zero or more whitespace
        characters (except newline), followed by a double-quote
        character; the backslash, newline, whitespace and double-quote
        character are then skipped, and parsing of the literal string
        continues after the double-quote character.

        \item Otherwise, the backslash character must begin an escape
        sequence. Escape sequences are:
        \begin{itemize}

            \item escape sequences that may appear in character constants;

            \item \verb|\x| followed by exactly two hexadecimal digits,
            standing for the byte whose value is expressed in hexadecimal
            by these two digits;

            \item \verb|\u| followed by exactly four hexadecimal digits,
            standing for the UTF-8 encoding of the code point whose value
            is expressed in hexadecimal by these four digits;

            \item \verb|\U| followed by exactly six hexadecimal digits,
            standing for the UTF-8 encoding of the code point whose value
            is expressed in hexadecimal by these six digits.

        \end{itemize}

        Note that hexadecimal digits are ASCII digits ``\verb|0|'' to
        ``\verb|9|'', uppercase letters ``\verb|A|'' to ``\verb|F|'',
        and lowercase letters ``\verb|a|'' to ``\verb|f|''. Case is not
        significant for hexadecimal digits. Unrecognized escape
        sequences trigger errors.

    \end{itemize}

\end{itemize}

For instance, this literal string:
\begin{verbatim}
    "Hello\
    " World!"
\end{verbatim}
uses a line escape and has contents ``Hello World!''.

The four following strings:
\begin{verbatim}
    "café"
    "caf\xc3\xa9"
    "caf\u00E9"
    "caf\U0000E9"
\end{verbatim}
all define the same sequence of five bytes. Take care that ``\verb|\x|''
escapes allow inclusion of arbitrary byte values which do not
necessarily correspond to the valid UTF-8 encoding of a sequence of code
points.

Apart from line escapes, newline characters may not appear into a
literal string (but a ``\verb|\n|'' escape sequence can be used to
include a newline character in the string contents). Thus, a literal
string may span several lines only if each line (except the last) ends
with a line escape. Moreover, a string literal must be terminated before
the end of the current text stream; string literals do not span across
files.

Since the whitespace characters that are part of a line escape do not
include the newline character, a comment is not possible within that
whitespace.
