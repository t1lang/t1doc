\section{Functions}

Every piece of code in T1 is a \emph{function}. A function has a name
(which is a character string) and is \emph{registered}; the registration
is what makes the function callable. Several functions may have the same
name, but will then differ by the types under which they are registered.

\subsection{Runtime Model}

Function parameters, and returned values, are exchanged on a
\emph{stack}. The stack contains only values, which are references.
Every function extracts the parameters it needs from the stack, and
pushes back its returned values on the stack. This inherently allows
functions to return several values.

A function, when invoked, has an \emph{activation context}, which is
traditionally called a \emph{stack frame}. This is disjoint from the
stack described above. Implementations may use a stack structure to
allocate activation contexts; in that case, that stack structure is
often called the ``system stack'', while the normal stack for values is
called the ``data stack''. Here, we will strive the avoid the confusion
by using the expression ``activation context'', and reserving the term
``stack'' for the data stack.

The activation context is a transient memory area that will contain the
local variables for the function, and may save the current execution
point for the function. When a function calls another function, the
current instruction pointer is saved in the function activation context,
and a new activation context is created for the called function; when
that called function returns, its activation context is released, and
the instruction pointer is restored from the activation context of the
caller. Exactly how this happens is an implementation detail.

Local variables are slots that can receive values; during translation of
the source code, local variables have names, which allows source code to
issue read and write instructions for these variables.

\begin{rationale}
In Forth, the system stack is explicit, with words \verb|R>| and
\verb|>R| to move values from the system to the data stack, and back.
Since some tasks may require more complicated data movements (the usual
example is adding tridimensional vectors together), Forth also defines
local variables, which are usually located on the system stack. Local
variable names are translated at compilation time into depths on the
system stack, which means that local variables don't interact well, or
at all, with facilities that access the system stack, such as explicit
words (\verb|R>| and its ilk...), or loop counters. Thus, a function
may use the system stack explicitly, \emph{or} use local variables,
but should not try to mix both.

For T1, which is not encumbered by compatibility with existing legacy
code, it seems simpler to avoid the complications and normalize on a
single system. Thus, local variables have been chosen, and the system
stack is not made visible to user code, except as the ``activation
context'' abstraction.
\end{rationale}

\emph{Locally allocated instances} are an extension of local variables:
these are object instances that are part of the activation context. This
corresponds to automatic variables in C or C++; arrays of references or
embedded structures can be obtained that way. These locally allocated
instances are nominally destroyed when the owner function exits. T1 does
not have destructors (in the C++ sense) or finalizers (in the Java
sense), thus the notion of ``destruction'' really means memory
deallocation. During interpretation, the garbage collector is used for
such instances, meaning that local allocation is not different from
normal heap allocation. In compiled code, allocation is really done in
the activation context, and has some restrictions so that memory safety
is maintained in all its facets (notably guaranteed maximum stack
growth): the size of such instances must be known at compile-time, and
instances shall not ``escape'' to outer contexts, i.e. remain reachable
once the owner function has returned.

\begin{rationale}
A typical use for stack allocation is creation of an array view instance
to designate a chunk of an array provided by a calling function, for
purposes of using that new array view instance as parameter to another
nested function. Such operations should be doable even when compiling
for targets that do not support dynamic memory allocation. Another use
is assembly of a small character string, e.g. for immediate display.
\end{rationale}

\subsection{Function Invocation}

The only way to invoke a function is by name. Function names may be
arbitrary; syntactically, a name token is used, and unqualified names
are translated to qualified names by the parser, thus most function
calls should use qualified names.

To be callable, a function must be \emph{registered}. A function
registration includes its name, and parameter types. For instance,
this code defines and registers a function:
\begin{verbatim}
    : foo (int string)
        # Here goes the function body
\end{verbatim}
If the current namespace is \verb|def|, then the function is registered
under the name \verb|def::foo| and with two parameter types,
\verb|std::int| and \verb|std::string|. The intent is that if some code
calls the function ``\verb|def::foo|'', and at that exact time, the
runtime types of the top two stack elements are \verb|std::int| and
\verb|std:string|, respectively, then the function defined above shall
be the one to be called. Types are provided in ``stack order'', i.e.
the rightmost element is the top-of-stack.

The function invocation process works in the following way:
\begin{itemize}

    \item The function invocation uses a specific name; only functions
    registered under that exact name are considered.

    \item A set of all \emph{matching functions} is defined: these are
    all the functions (with the correct invocation name) for which the
    registered parameter types match the runtime types of the
    corresponding stack elements at call time. E.g. in the example
    above, that function \verb|def::foo| is a matching function if and
    only if the top stack element has type \verb|std::string| or a
    sub-type thereof, and the stack element immediately below has type
    \verb|std::int| or a sub-type thereof.\footnote{In that specific
    case, \texttt{\textbf{std::int}} cannot have sub-types, but
    \texttt{\textbf{std::string}} can.}

    \item The matching functions are ordered by \emph{precision}. Let
    $f$ be a function registered with parameter types $r_m, r_{m-1},...
    r_1$ (in stack order, $r_1$ is top-of-stack), and $g$ be a function
    registered with parameter types $s_n, s_{n-1},... s_1$. $f$ will be
    said to be \emph{more precise} than $g$ if and only if all of the
    following hold:
    \begin{itemize}

        \item $m \ge n$ (i.e. $f$ is registered with at least as many
        parameter types as $g$)

        \item For all $1\le i\le n$, type $r_i$ is a sub-type of $s_i$.

    \end{itemize}

    \item If one of the matching functions is more precise than all other
    matching functions, then that function is called. Otherwise, an
    error occurs.

\end{itemize}

The following important points must be noticed:
\begin{itemize}

    \item Precision order is partial. Two given functions are not
    necessarily comparable, i.e. neither being ``more precise'' than
    the other. The invocation process does not require that all matching
    functions be comparable to each other, but that one can be compared
    to all others, and found to be more precise than all others.

    \item A failure will be reported if there is no matching function,
    but also if there are several and none is more precise than all the
    others.

    \item Since sub-typing is acyclic (except that every type is deemed
    to be a sub-type of itself), the only way for two functions $f$ and
    $g$ to be such that $f$ is more precise than $g$ and $g$ is more
    precise than $f$ at the same time, is to have $f$ and $g$ registered
    with the exact same parameter types. This situation is explicitly
    forbidden: any attempt at registering a function with the same name
    and parameter types as an already registered function triggers an
    error.

    \item If a function is registered with $n$ parameter types, and the
    stack contains fewer than $n$ elements at call time, then that
    function is not a matching function.

    \item The parameter types used for registration do not necessarily
    exhaust all the actual function parameters. A function registered
    with two parameter types may use more than two parameters. Moreover,
    registration says nothing about how the number and types of values a
    function may return (i.e. leave on the stack when exiting).

\end{itemize}

This process works best when registered functions use the same patterns.
For instance, it is expected that most functions in a given application
will work like ordinary methods as in classic OOP, i.e. be dispatched
based on the type of a single parameter, which will be ``the object on
which the method is invoked''. To allow functions to be used without
undue collisions, even if the same names are used, it is best if all
such method-like functions are registered such that the owner object is
the top-of-stack (i.e. rightmost parameter in the list).

\subsection{Immediate Functions}

An \emph{immediate function} is a function which is registered with no
parameter types, and a special ``immediate'' flag. The role of immediate
functions is to be invoked as soon as they are encountered in the source
code, during interpretation; this is how additional syntax is defined.
